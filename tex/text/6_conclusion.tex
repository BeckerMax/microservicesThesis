\chapter{Conclusion}
\label{con}
%TODO a little to retrospective

\section{Summary}
Prior work on the topic of the microservice architectural style mainly covers general overviews \cite{Newman2015,Wolff2016}. 
As there is no common definition of the relatively new term microservices, different sources draw different, coarse pictures of its characteristics.
Many refer to the successful flagship implementations in the industry, like Netflix or Amazon.
But, there is no thorough trade-off discussion of microservices, as well as no explicit decision guidance for which application scenarios to use microservices known to the author of the thesis.

The thesis used quality attribute scenarios to systematically discuss trade-offs of microservices.
A list of quality scenarios relevant for the evaluation of microservices was created (Section~\ref{qua:architecturalQua}).
Each quality scenario was assessed and discussed in regard to whether microservice or the three-tier architectural style is more suitable to achieve the scenario (Section~\ref{quaMicro:rating}).
In addition, organizational factors and their influence on using microservices were discussed (Section~\ref{quaMicro:tradeoffOrgRequirements}).
This evaluation was based on research from literature, conference talks and analysis of industry examples of microservice architectures. 
A subset of the scenarios was empirically evaluated by interviewing microservice practitioners through a questionnaire (Chapter~\ref{que:empiricalEvaluation}).

During the evaluation of the quality attributes it became clear that giving a general guidance in which application scenarios to use microservices cannot be achieved by this thesis.
Firstly, a high-level description of an application scenario is too vague to be reliably assessed regarding quality attributes. 
Secondly, the decision which architectural style to use is dependent on many factors (quality attributes are just one of them) and they cannot all be covered here.
% in order to give a realistic decision guidance.
Trade-offs on several layers contribute to the decision of using microservices or not.

However, what this thesis presents is a discussion of different trade-offs of microservices in comparison to more monolithic styles, like a three-tier architecture.
Not a definite decision guideline, but a discussion starter is presented based on the condensed knowledge of several microservice practitioners.
In addition to quality attributes, also organizational factors are discussed, which play a major role in the decision for or against microservices.

Summarizing the findings from the quality rating in Section~\ref{quaRating:conclusion} microservices should be considered for the following use cases (related quality scenario numbers in brackets):

\begin{itemize}
\item High development velocity, meaning continuous innovation in large projects consisting of several teams (\hyperref[quaMicro:s16]{16} and organizational scenarios \hyperref[quaMicro:so1]{1}, \hyperref[quaMicro:so2]{2}).
\item High availability for large scale distributed systems, especially if eventual consistency can be accepted (\hyperref[quaMicro:s12]{12}, \hyperref[quaMicro:s13]{13}).
\item Excellent utilization of on-demand hardware through fine-grained auto-scaling of load. For example \textit{pay for service} (\hyperref[quaMicro:s2]{2}, \hyperref[quaMicro:s4]{4}, \hyperref[quaMicro:s22]{22}). 
\end{itemize}

On the contrary, one should rather not consider microservices if:

\begin{itemize}
\item The domain of the project is not fully known (see \textit{\hyperref[quaMicro:modifiabilityDiscussion]{modifiability discussion}} in \ref{quaMicro:modifiabilityDiscussion}).
\item The organization does not have or is not ready to heavily invest in competence to develop and operate complex, distributed systems (\hyperref[quaMicro:s9]{9}, \hyperref[quaMicro:s10]{10} and see \textit{\hyperref[quaMicro:distributedSystemDownsides]{distributed system discussion}} in \ref{quaMicro:distributedSystemDownsides}).
\item ACID conform transactions spanning over multiple business areas are needed (\hyperref[quaMicro:s11]{11}).
\end{itemize}

For most other quality attributes, microservices in comparison to three-tier architectures have up- and downsides.
Section~\ref{quaMicro:distributedSystemDownsides} holds a discussion of these trade-offs ordered by quality attributes.
An empirical evaluation through a questionnaire support the results of this evaluation (Section~\ref{res:overview}).

\section{Outlook and future work}
Microservices are another tool in the architectural toolkit that need to be chosen appropriately.
They will not replace monolithic applications in the near future.
Instead, they present themselves as an alternative to other architectural styles.
To conclude the thesis, this section discusses promising fields for future work on microservices.

\paragraph{Adoption of microservices}
At this moment it is too early to provide a final judgment of how effective microservices actually are. 
Most microservices applications are still rather young and how modifiable a system is will be evident after years.
As Fowler states, it is likely that the success of microservices is dependent on many factors in the surrounding environment, both political and technological \cite{FowlerMSOverview}.
For example, the skill of developers to handle the challenges of a distributed system might play an important role in how successful the style will be in a concrete use case.

Because most microservices-based applications are still rather young this thesis comes with a caveat. 
It uses information from the early adopters of microservices, which are probably not representative for all companies.
Only the future will tell whether the later adopters will be successful with microservices.
This will decide whether microservices can emerge from being a hyped topic to becoming an established and widely adopted architectural style.
Future work will have to constantly validate the currently known trade-offs of microservices with the growing evidence coming from teams sharing their experiences.
Also researching anti-patterns from failed microservice adoptions would be interesting.

\paragraph{Organizational implications of microservices}
This thesis offers a glimpse on the organizational implications of microservices.
Apparently, system design is inherently connected with the organizational structure of a company \citep[p. 201]{Newman2015}.
For example, microservices are frequently associated with the inverse Conway Maneuver (for Conways Law see \ref{bac:organisationalCharacteristics}).
The idea is to let the system design influence the team structure, instead of the other way around \citep[p. 201]{Newman2015}.
For microservices, this would mean that the teams follow the separation of services into business capabilities.
Reorganizing teams in an established organization into small independent teams is a large investment and future work could give examples and provide support on how to do that.

Another team related concept touched in this thesis is cross-functional teams.
Following the Amazon.com example, a microservice should be built and operated by one single team.
This team is ``completely responsible for the service -- from scoping out the functionality, to architecting it, to building it, and operating it \cite{Vogels2006}''.
This underlines the point that the decision for microservices has implications for team structure and organization.
It seems that a discussion about the organizational implications of microservices is underrepresented in the current literature, especially in relation to its importance.
Further researching the organizational implications of microservices seems to be interesting and highly relevant.

\paragraph{Patterns and strategies for implementing microservices}
With microservices, the complexity rises in fields like operation, deployment or programming.
The fact that microservices decouple the lifecycles of components enables innovation but acts as a complexity booster \cite{FowlerTradeoffsDistribution2015}.
Certain patterns and strategies are described as de facto standards for microservice architectures.
A few of them are:
\begin{itemize}
\item Tactics for reliability, like Bulkhead
\item Automated deployment with \ac{CD} and automated infrastructure provisioning
\item Sophisticated monitoring and logging setups
\item Simulating failure in production
\item Service interface versioning strategies
\item Patterns to implement network communication between services
\item Strategies to deal with partial failures and indeterminacy in a distributed system \cite{Kendall1994}.
\end{itemize}
These strategies did not necessarily evolve with microservices but they are inherently important for them, or arguably any large scale distributed system.
An approach that researches these strategies and patterns in order to give concrete guidance on how to handle the complexity and challenges of microservices appears useful and promising.

% Kerniger Schlusssaz, ein kerniges Statement zu meinen Erkenntnissen und oder was weitere Forschung hinzufügen kann.