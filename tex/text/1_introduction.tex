\chapter{Introduction}

\section{Motivation}
\label{intro:motivation}
\begin{chapquote}{Marc Andreessen, Wall Street Journal, 2011}
``Software is eating the world.''
\end{chapquote}

In recent years, formerly stable industries have been disrupted by companies with software at their core.
The worlds largest bookseller is Amazon.
The worlds largest video service is Netflix.
The worlds largest direct marketing platform is Google.
As a result, companies that formerly had their core business outside of software now need to become software companies to stay competitive. \cite{Andreessen2011}

The growing importance of software has lead to new requirements for software systems.
Speed of innovation and continuous availability on the internet have become competitive advantages for numerous business fields.
Cloud computing, defined as on-demand network access to computing resources, emerged as an answer to some of these new requirements \cite{DefCC2011}. 

The need to innovate large web scale systems faster has lead to \textit{microservices}, an emerging architectural concept that decouples the life-cycle of components in a software system.
Microservices rely on a conglomerate of software technologies and organizational and software-development practices that enables small teams responsible for single components or business units to innovate independently from the remainder of large systems.

There are several books, talks and articles concerning the definition and characteristics of microservices.
Mainly, the existing definitions are based on successful flagship implementations in the industry.
Also, voices of reason are slowly becoming louder as some teams got burned by being too eager in embracing microservices and stumbling over the complexity therein \cite{FowlerMSPremium2014}.

Even with the available resources, a question arises: are microservices a good choice for the system you are working on?
Fowler answers with ``it depends'' \cite{FowlerMSPremium2014}.
And the same does this thesis.
What this thesis contributes, however, is an investigation of what it \textit{depends on} whether microservices are a good architectural choice for one case.

\section{Research goal}
\label{intro:goal}

Measured by its recent popularity, microservices appear to be a promising architectural style.
Multiple large web-based applications such as Amazon and Netflix successfully use it in production today.
However, such amazingly popular and large-scale applications are not necessarily the proof that microservices are fit for every use case.
Thus, whether microservices are the right style for a concrete business scenario in a given organization is an extremely compelling question.
This thesis aims at demystifying microservices by systematically discussing their trade-offs, which leads to the problem definition:

\begin{addmargin}[2em]{2em}
Given a concrete application use case, how to decide whether the application is better built with microservices or with an alternative, possibly more monolithic style?
\end{addmargin}

This problem definition relies on defining how to evaluate software architectures in general.
Under the assumptions that \citep[p. 40]{Bass2012}:
\begin{enumerate}
\item An architecture will inhibit or enable the achievement of a system's quality attributes.
\item One can predict many aspects of a system's qualities by studying its architecture.
\end{enumerate}
This thesis is built on the assumption that, since one can use quality attributes to assess software architectures, the same are applicable to microservice architectures as well.
A challenge with quality attributes is that they are usually created for a specific business or application scenario.
Thus, one of the goal of this thesis is to evaluate whether quality attributes can be used to provide general architectural guidance.

Embracing the problems outlines above, this thesis aims at accomplishing the following main goal:
\begin{addmargin}[2em]{2em}
\textit{To provide guidance in terms of under which circumstances, including quality as well as organizational requirements, microservices are a well suited -- or undesirable -- architectural choice.}
\end{addmargin}

\section{Research questions}
\label{intro:questions}
The problem statement outlined in Section \ref{intro:goal} can be broken down in the following research questions:

\paragraph{What are the main characteristics of microservices?}
Is there a general definition of microservices?
What are the main technical characteristics?
What are the main organizational characteristics?
How do microservices compare to more monolithic architectures?
%What are prerequisites

\paragraph{How well can quality attributes of different categories be achieved in systems built out of microservices?}
What is a quality attribute in the context of software architecture?
What is a way to systematically find and express quality attributes for software systems?
How do microservices compare to more monolithic architectural styles, like a 3-tier architecture in regard to selected quality attributes?

\paragraph{Which software systems are particularly well suited to be build with microservices?}
Which application scenarios benefit from being built with microservices -- and which not?
What are the technical and organizational challenges when implementing and operating a microservice architecture?
Is it possible to give a general decision guidance for when to use microservices?

\section{Research approach}
\label{ther}
The approach for answering the research questions presented in Section~\ref{intro:questions} is discussed here.

\textit{Question}: What are the main characteristics of microservices?

\textit{Approach}:
In order to better understand the trade-offs of microservices, it is necessary to investigate the characteristics.
As microservices are a relatively new topic, the thesis will not solemnly rely on literature research but also use conference talks, blog entries and podcasts to form a definition.
In addition current microservice based software systems, like the Amazon online shop or Netflix are analyzed.

\textit{Question}: How well can quality attributes of different categories be achieved in systems built out of microservices?

\textit{Approach}:
This thesis adopts quality attribute scenarios to evaluate for which software systems microservices are a fitting choice.
Quality attribute scenarios emerged before microservices were popularized, and are used for assessing whether a software system will have a set of desired qualities \cite{BarbacciQualityAttribute2003}.
Using the ISO standard 25010, relevant quality attributes are systematically listed and written down as scenarios.
The scenarios are evaluated by researching literature and other sources, like conference talks or articles and putting this information into context. 
In addition, an empirical evaluation is presented, which uses a questionnaire for evaluation of quality scenarios.

\textit{Question}: Which software systems are particularly well-suited to be build with microservices?

\textit{Approach}:
Through the discussion of quality attributes a general recommendation is given for which quality requirements microservices are suitable.
Also organizational factors are discussed and their influence on the decision of using microservices.
In addition, the questionnaire is designed so that participants order quality attributes by importance for different application scenarios.
Then, by mapping quality attributes, guidance should be given in which of the selected application scenarios microservices are a well-suited approach.
%Guidance should be given, whether microservices are a favorable approach to achieve the qualities, which are important for the application scenario.

\section{Thesis structure}

Chapter~\ref{bac} defines microservices on a technical and organizational level.

Chapter~\ref{qua:architecturalQua} discusses quality attribute scenarios in general and uses a structured approach to create relevant quality attribute scenarios for the evaluation of microservices.

Chapter~\ref{qua:qualitiesRating} presents the main discussion of this thesis.
The given quality scenarios were rated based on literature research, which leads to a trade-off discussion of microservices on technical and organizational levels.

Chapter~\ref{que:empiricalEvaluation} presents an empirical evaluation of the results from Chapter~\ref{qua:qualitiesRating}.
This chapter also introduces a questionnaire which was used to have microservice practitioners rate a subset of the quality scenarios rated in Chapter~\ref{qua:qualitiesRating}. 
Section~\ref{res:overview} discusses the outcomes of the questionnaire rating and the literature-based rating are compared.

Chapter~\ref{con} concludes this thesis by reflecting on the research goals and discussing future work.
